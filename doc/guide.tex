\documentclass[a4paper]{report}

\usepackage[margin=3.0cm]{geometry}
\usepackage{amsmath}
\usepackage[pdftex]{graphicx}
%\usepackage{graphics}
\usepackage{subfig}



\title{HPIPM reference guide}
\author{Gianluca Frison}



\begin{document}

\maketitle
\tableofcontents





%%%%%%%%%%%%%%%%%%%%%%%%%%%%%%%%
\chapter{Introduction}
%%%%%%%%%%%%%%%%%%%%%%%%%%%%%%%%

HPIPM - High-Performance Interior Point Method.

HPIPM is a library providing a collection of quadratic programs (QP) and routines to manage them.
Aim of the library is to provide both stand-alone IPM solvers for the QPs and the building blocks for more complex optimization algorithms.

At the moment, three QPs types are provided: dense QPs, optimal control problem (OCP) QPs, and tree-structured OCP QPs.
These QPs are defined using C structures.
HPIPM provides routines to manage the QPs, and to convert between them.

HPIPM is written entirely in C, and it builds on top of BLASFEO~\cite{BLASFEO}, that provides high-performance implementations of basic linear algebra (LA) routines, optimized for matrices of moderate size (as common in embedded optimization).



%%%%%%%%%%%%%%%%%%%%%%%%%%%%%%%%
\chapter{Dense QP}
%%%%%%%%%%%%%%%%%%%%%%%%%%%%%%%%

The dense QP is a QP in the form
\begin{align*}
\min_{v,s} & \quad \frac 1 2 \begin{bmatrix} v \\ 1 \end{bmatrix}^T \begin{bmatrix} H & g \\ g^T & 0 \end{bmatrix} \begin{bmatrix} v \\ 1 \end{bmatrix} + \frac 1 2 \begin{bmatrix} s^l \\ s^u \\ 1 \end{bmatrix}^T \begin{bmatrix} Z^l & 0 & z^l \\ 0 & Z^u & z^u \\ (z^l)^T & (z^u)^T & 0 \end{bmatrix} \begin{bmatrix} s^l \\ s^u \\ 1 \end{bmatrix} \\
s.t. & \quad A v = b \\
& \quad \begin{bmatrix} \underline v \\ \underline d \end{bmatrix} \leq \begin{bmatrix} J_{b,v} \\ C \end{bmatrix} v + \begin{bmatrix} J_{s,v} \\ J_{s,g} \end{bmatrix} s^l \\
& \quad \begin{bmatrix} J_{b,v} \\ C_n \end{bmatrix} v - \begin{bmatrix} J_{s,v} \\ J_{s,g} \end{bmatrix} s^u \leq \begin{bmatrix} \overline v \\ \overline d \end{bmatrix} \\
& \quad s^l\geq 0 \\
& \quad s^u\geq 0
\end{align*}
where $v$ are the primal variables, $s^l$ ($s^u$) are the slack variables of the soft lower (upper) constraints.
The matrices $J_{\dots}$ are made of rows from identity matrices.
Furthermore, note that the constraint matrix with respect to $v$ is the same for the upper and the lower constraints.


%%%%%%%%%%%%%%%%%%%%%%%%%%%%%%%%
\chapter{OCP QP}
%%%%%%%%%%%%%%%%%%%%%%%%%%%%%%%%

The OCP QP is a QP in the form
\begin{align*}
\min_{x,u,s} & \quad \sum_{n=0}^N \frac 1 2 \begin{bmatrix} u_n \\ x_n \\ 1 \end{bmatrix}^T \begin{bmatrix} R_n & S_n & r_n \\ S_n^T & Q_n & q_n \\ r_n^T & q_n^T & 0 \end{bmatrix} \begin{bmatrix} u_n \\ x_n \\ 1 \end{bmatrix} + \frac 1 2 \begin{bmatrix} s^l_n \\ s^u_n \\ 1 \end{bmatrix}^T \begin{bmatrix} Z^l_n & 0 & z^l_n \\ 0 & Z^u_n & z^u_n \\ (z^l_n)^T & (z^u_n)^T & 0 \end{bmatrix} \begin{bmatrix} s^l_n \\ s^u_n \\ 1 \end{bmatrix} \\
s.t. & \quad x_{n+1} = A_n x_n + B_n u_n + b_n & \quad n=0,\dots,N-1 \\
& \quad \begin{bmatrix} \underline u_n \\ \underline x_n \\ \underline d_n \end{bmatrix} \leq \begin{bmatrix} J_{b,u,n} & 0 \\ 0 & J_{b,x,n} \\ D_n & C_n \end{bmatrix} \begin{bmatrix} u_n \\ x_n \end{bmatrix} + \begin{bmatrix} J_{s,u,n} \\ J_{s,x,n} \\ J_{s,g,n} \end{bmatrix} s^l_n & \quad n=0.\dots,N \\
& \quad \begin{bmatrix} J_{b,u,n} & 0 \\ 0 & J_{b,x,n} \\ D_n & C_n \end{bmatrix} \begin{bmatrix} u_n \\ x_n \end{bmatrix} - \begin{bmatrix} J_{s,u,n} \\ J_{s,x,n} \\ J_{s,g,n} \end{bmatrix} s^u_n \leq \begin{bmatrix} \overline u_n \\ \overline x_n \\ \overline d_n \end{bmatrix} & \quad n=0.\dots,N \\
& \quad s^l_n\geq \underline{s}^l_n & \quad n=0.\dots,N \\
& \quad s^u_n\geq \underline{s}^u_n & \quad n=0.\dots,N
\end{align*}
where $u_n$ are the control inputs, $x_n$ are the states, $s^l_n$ ($s^u_n$) are the slack variables of the soft lower (upper) constraints
and $\underline{s}^l_n$ and $\underline{s}^u_n$ are the lower bounds on lower and upper slacks, respectively.
The matrices $J_{\dots,n}$ are made of rows from identity matrices.
Note that all quantities can vary stage-wise.
Furthermore, note that the constraint matrix with respect to $u$ and $x$ is the same for the upper and the lower constraints.

\begin{verbatim}
int d_memsize_ocp_qp(int N, int *nx, int *nu, int *nb, int *ng, int *ns);
\end{verbatim}

\begin{verbatim}
void d_create_ocp_qp(int N, int *nx, int *nu, int *nb, int *ng, int *ns, 
    struct d_ocp_qp *qp, void *memory);
\end{verbatim}

\begin{verbatim}
void d_cvt_colmaj_to_ocp_qp(double **A, double **B, double **b, 
    double **Q, double **S, double **R, double **q, double **r, 
    int **idxb, double **lb, double **ub, 
    double **C, double **D, double **lg, double **ug, 
    double **Zl, double **Zu, double **zl, double **zu, int **idxs, 
    double **ls, double **us,
    struct d_ocp_qp *qp);
\end{verbatim}



%%%%%%%%%%%%%%%%%%%%%%%%%%%%%%%%
% \chapter{Tree OCP QP}
%%%%%%%%%%%%%%%%%%%%%%%%%%%%%%%%



\end{document}
